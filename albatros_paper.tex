%%%%%%%%%%%%%%%%%%%%%%%%%%%%%%%%%%%%%%%%%%%%%%%%%%%%%%%%%%%%%%%%%%%%%%%%%%%%
%% Trim Size : 11in x 8.5in
%% Text Area : 9.6in (include Runningheads) x 7in
%% ws-jai.tex, 26 April 2012
%% Tex file to use with ws-jai.cls written in Latex2E.
%% The content, structure, format and layout of this style file is the
%% property of World Scientific Publishing Co. Pte. Ltd.
%%%%%%%%%%%%%%%%%%%%%%%%%%%%%%%%%%%%%%%%%%%%%%%%%%%%%%%%%%%%%%%%%%%%%%%%%%%%
%%

%\documentclass[draft]{ws-jai}
\documentclass{ws-jai}
\usepackage[flushleft]{threeparttable}
\usepackage{siunitx}
\usepackage{amsmath}
\usepackage{gensymb}
\usepackage[colorlinks=true,allcolors=blue]{hyperref}
\usepackage{graphicx}
\usepackage{caption}
\usepackage{subcaption}
\usepackage[bottom]{footmisc}
\usepackage{url}

% Define stuff here
\def\albatros{ALBATROS}

\begin{document}

\catchline{}{}{}{}{} % Publisher's Area please ignore

\markboth{Author's Name}{ALBATROS instrument}

\title{The Array of Long-Baseline Antennas for Taking Radio
  Observations from the Sub-Antarctic}

\author{First Author$^{2}$, Second Author$^{3}$, Third Author$^{3}$ and Fourth Author$^{4}$}

\address{
$^{2}$Department, University Name, City, State ZIP/Zone, Country, fauthor@university.com\\
$^{3}$Group, Company, Address, City, State ZIP/Zone, Country\\
$^{4}$Group, Company, Address, City, State ZIP/Zone, Country, fauthor@company.com
}

\maketitle

\corres{$^{2}$Corresponding author.}

\begin{history}
\received{(to be inserted by publisher)};
\revised{(to be inserted by publisher)};
\accepted{(to be inserted by publisher)};
\end{history}

\begin{abstract}
The low frequency radio astronomy has the highest potential in discovering the history of the Universe, this includes observations of the first stars and the mapping of dark ages. The Array of Long Baseline Antennas for Taking Radio Observations from the Sub-Antarctic (ALBATROS) is a new interferometric array. It consists of autonomous antenna stations that will map the low-frequency sky from Marion Island. One autonomous station was deployed in Marion Island in April 2019. The operating frequency range is \SIrange{1.2}{81}{\MHz} with baselines of $\approx \SI{20}{\km}$. A two element inteferometer, the ALBATROS - Exploratory Gizmo on the Ground (ALBATROS-EGG) was deployed in Marion Island in April 2018. \\
\end{abstract}

\keywords{cosmology: observations; dark ages; instrumentation: interferometers}

\section{Introduction}

Measurements of redshifted \SI{21}{\cm} emission of neutral hydrogen
across a wide range of radio frequencies have the potential to
elucidate the universe's history from the cosmic ``dark ages'' up to
the formation of large-scale structures (see,
e.g.,~\citep{ska_physics,2013PhRvD..87d3002L,2014ApJ...782...66P}.
The dark ages, which occurred after recombination and correspond to a
period when the universe was filled with neutral hydrogen, are
unexplored to date and represent one of the final observational
frontiers in cosmology.  This epoch contains a potential wealth of
cosmological information~\citep{loeb_zaldarriaga, 2019arXiv190710853C,
  2019arXiv190804296K}, but the required redshifted frequencies of
$\lesssim 30$~MHz are exceptionally difficult to observe.  The primary
experimental challenges include Galactic foreground emission,
ionospheric interference, and terrestrial radio-frequency interference
(RFI).

Very few experiments have measured the radio sky at $\lesssim 30$~MHz.
Comprehensive reviews exist elsewhere ({\bf citation?}), and here, we
highlight only a few specific examples.  At the very lowest
frequencies, the state of the art among ground-based measurements
dates from the 1950s, when Reber and Ellis caught brief glimpses of
the $2.1$~MHz sky at $\sim 5$\degree\ resolution, using an array of
192~dipoles~\citep{reber, article, 1988A&A...195..372W}.  A few
space-based missions have also performed measurements at similarly low
frequency ranges; for example, the Radio Astronomy Explorer-2 operated
at \SIrange{0.025}{13}{\MHz} with $\sim 10$\degree\ resolution at
\SI{4.7}{\MHz}~\citep{1975A&A....40..365A}.  Measurements with
resolution finer than few-degree scales exist only at higher
frequencies.  For example, the Dominion Radio Astrophysical
Observatory surveyed the northern sky at \SI{22}{\MHz} with
\SIrange{1.1}{1.7}{\degree} resolution~\citep{1999A&AS..137....7R},
and most recently, the Owens Valley Long Wavelength Array mapped the
sky with \SI{15}{\arcminute} resolution between 36.528~MHz and
73.152~MHz~\citep{2018AJ....156...32E}.  Although this experimental
list is not comprehensive, it does illustrate the dearth of
information about the $\lesssim 30$~MHz sky and the lack of high
resolution measurements at the lowest frequencies.

Preliminary observations from Marion
Island~\citep{2019JAI.....850004P} suggest that despite the
present-day RFI environment, low-frequency observations may still be
accessible from carefully selected locations and with new technology
developments.  In this paper, we present the Array of Long Baseline
Antennas for Taking Radio Observations from the Sub-Antarctic
(\albatros), a new experimental effort that aims to map the
low-frequency sky using an array of autonomous antenna stations.  We
describe the overall instrument design and preliminary measurements
from engineering runs that were performed on Marion Island during
2018--2019.

\section{ALBATROS overview}

{\bf Write text here about the overall final configuration.  Describe
  the site, show synthesized beam, discuss high level system
  requirements (power, data storage, etc).  Include deployment
  schedule and introduce 2-element pathfinder in the following
  section.}

The final array will consist of $\approx$ 10 antennas operating at
1.2-81 MHz with baselines up to 20 km. A two-element pathfinder was
deployed in April 2018, the first autonomous station was deployed in
April 2019.

\section{Two-element pathfinder}

\begin{figure}
  \begin{center}
    \includegraphics[width=0.7\linewidth]{Figures/ALBATROS-EGG.PNG}
    \caption{The two-element \albatros\ pathfinder on Marion Island.
      Two dual-polarization LWA antennas are separated by roughly
      110~m on an east--west baseline. Coaxial cables connect the
      antennas to an orange shipping container that houses the readout
      electronics and serves as our ``command module.''}
    \label{Fig:albatros2}
  \end{center}
\end{figure}

\begin{figure}
  \begin{center}
    \includegraphics[width=1.0\linewidth]{Figures/albatros_2elem_schematic/albatros_2elem_schematic.pdf}
    \caption{Schematic for the two-element \albatros\ pathfinder.
      Signals from two dual-polarzation LWA antennas are amplified by
      front-end active baluns ({\bf citation for FEE}), and 100-m
      coaxial cables connect the antennas to the back-end readout
      electronics, which are housed in a Faraday cage denoted by the
      dashed box.  Each of the four antenna outputs is passed to a
      second-stage electronics chain consisting of a bias tee,
      amplifier, and high- and low-pass filters.  The signals are
      digitized at 250~Msamp/s by a SNAP board, and the on-board FPGA
      is programmed with firmware to perform a full cross-correlation
      of all four inputs.  A Raspberry Pi controls the SNAP board and
      also saves the computed spectra to an SD card.  Power to the
      system is provided by a bank of 12~V batteries and several
      regulated voltage outputs.}
    \label{Fig:albatros2_schem}
  \end{center}
\end{figure}

The ALBATROS-EGG is a two element inteferometer which is making
exploratory measurements at a frequency range of \SI{1.2}{MHz}-
\SI{81}{MHz} separated by a baseline of \SI{110}{m} as shown in
\autoref{Fig:albatros2}. It was deployed in Marion Island in April
2018. Its schematic is shown in \autoref{Fig:albatros2_schem}. The
system uses the dual polarisation dipole like antennas which are
omidirectional patterned. The front end electonics (FEE) are
configured in a printed circuit board (PCB) which makes it easy for
them to be mounted on the supporting structure of the antennas. From
the FEE PCB, a coaxial cable sends the signal through to the bias tee
which extracts both the RF and the DC without any degradation.

The high pass and the low pass filter rejects the low frequency
signals and high frequency signals respectively. The signal is then
amplified by a $\approx$20 dB amplifier which then send the amplified
signals to the Smart Network ADC Processor (SNAP) board which does the
readouts and does a series of processing step. The raspberry pi
interacts with the SNAP board for data storage.

The final project (ALBATROS) will consist of autonomous antenna stations that will map the low frequency sky. Since these experiments are exploratory, they are taking steps towards achieving the future objective of probing the Dark Ages.The ALBATROS stations (huts) will be separated by baselines of \SI{\approx {20}}{km} as shown in \autoref{Fig:Marion}. The two-element interferomentric pathfinder is not yet operating autonomously, it uses the direct cross correlation technique whereas the yet to be ALBATROS will write the lowest 10 - 20 MHz baseband to disk then gets correlated afterwards. One ALBATROS fully autonomous station was deployed in Marion Island in April 2019 as shown in \autoref{Fig:autonomous}. \\

The single antenna analog signal chain is shown in \autoref{Fig:Signal Chain}. The components of the system are discussed in detail as per the block diagram illustrated in \autoref{Fig:Signal Chain}. This analog signal chain is an illustration of how the first autonomous station is configured. There might be changes to the block diagram at a later stage should there be a need to revise it for the stations.

\begin{figure}[h]
	\begin{center}
		\includegraphics[width=0.7\linewidth]{Figures/Signal-Chain.png}
		\caption{Analog signal chain block diagram for the newly deployed autonomous station the ALBATROS}
		\label{Fig:Signal Chain}
	\end{center}
\end{figure}

\subsection{Antenna}	
The system uses the dual polarisation dipole like antennas which are omidirectionally patterned. The dipole like antennas are of high preference for this project because they are relatively simple and they are omnidirectionally patterned \cite{Memo28}.

\subsection{Front End Electronics (FEE)}
All the front end components were incorporated for in a double sided printed circuit board (PCB) as shown in \autoref{Fig:Balun} and the block diagram is shown in \autoref{Fig:Balun Schematic}. The Monolithic Microwave Integrated Circuits (MMICs) is the used design for the PCB. One side of the PCB is populated with components and the other side is a solid copper ground plane aperiodically stitched to the grounded copper on the side populated with components. The receiver system is made up of the active balun, filter and the gain stage that connects to the \SI{100}{\metre} coaxial cable which is connected to the back end \cite{2012PASP..124.1090H}.\\ 
The input impedance (Z\textsubscript{o}) of \SI{50}{\ohm} is introduced to the dipole by the active balun. The signal is then fed through an amplifier which amplifies it by \SI{+24}{\decibel} of gain. The balanced signal is then converted to unbalanced through a 180 \degree hybrid coupler. The band pass filter (BPF) receives the single ended signal in order for it to reject all the frequencies which are not within the range of interest. The signal gets fed to a second amplifier which again amplies it by \SI{+24}{\decibel} of gain and the output impedance of the FEE is matched to a \SI{50}{\ohm} coaxial cable. The bias tee is responsible for providing power to the FEE and extracts the RF signal by the use of the coaxial cable. This unit has an overall gain of $\approx$ \SI{35}{\decibel} and an overall noise figure of $\approx$ \SI{2.7}{\decibel} to $\approx$ \SI{2.9}{\decibel} \cite{Memo35}.

\begin{figure}[h]
	\begin{center}
		\includegraphics[width=0.5\linewidth]{Figures/balun.jpg}
		\caption{Unenclosed FEE mounted on the ALBATROS-EGG antenna supporting structure.} 
		\label{Fig:Balun}
	\end{center}
\end{figure}

\begin{figure}[h]
	\begin{center}
		\includegraphics[width=0.7\linewidth]{Figures/Balun_Block.png}
		\caption{One Polarisation Block Diagram of the FEE \cite{2012PASP..124.1090H}}
		\label{Fig:Balun Schematic}
	\end{center}
\end{figure}

\subsection{Back End Electronics}	
The back end electronics consists of the bias tee, high pass filter (HPF), low pass filter (LPF) and the Smart Network ADC Processor (SNAP) board. \\

\textbf{Bias Tee} - It is used to extract the RF signal from a \SI{100}{\metre} LMR400 coaxial cable which has a nominal attenuation of \SI{\approx 0.4}{\decibel/\SI{100}{m}} - \SI{\approx 3.7}{dB/\SI{100}{\metre}} at \SI{1.2}{MHz} - \SI{81}{MHz} respectively \footnote{https://www.timesmicrowave.com/documents/resources/LMR-400.pdf}. The bias tee which is used is the Mini Circuits device ZFBT-4R2GW-FT+ which operates between \SI{0.14}{MHz} - \SI{4200}{MHz}. It has an insertion loss of \SI{0.16}{\decibel} at centre frequency of  \SI{\approx 10}{MHz} as per the datasheet \footnote{https://www.minicircuits.com/pdfs/ZFBT-4R2GW-FT+.pdf}.\\

\textbf{HPF} - The HPF rejects any signals with low frequencies and allow through signals with high frequencies \footnote{http://www.learningaboutelectronics.com/Articles/High-pass-filter.php}. The HPF used is the ZFHP-1R2+ Mini Circuits device which operates between  \SI{1.2}{MHz} - \SI{800}{MHz}. It has a nominal insertion loss of $\approx$ \SI{0.2}{\decibel} at centre frequency of  \SI{\approx10}{MHz} as per the datasheet \footnote{https://www.minicircuits.com/pdfs/ZFHP-1R2+.pdf}.\\

\textbf{LPF} - The LPF rejects all signals with high frequencies and allows all signals with low frequencies \footnote{http://www.learningaboutelectronics.com/Articles/Low-pass-filter.php}. The LPF used is the SLP-90+ Mini Circuits device which operates between \SI{1}{MHz} - \SI{400}{MHz}. It has a nominal insertion loss of $\approx$ \SI{0.14}{\decibel} at the centre frequency of  \SI{\approx 10}{MHz} \footnote{https://www.minicircuits.com/pdfs/SLP-90+.pdf}.\\

\textbf{Amplifier} - The amplifier at the backend amplifies the signal which is sent through to the snapboard by the gain of \SI{+20}{\decibel}. The amplifier used is the Mini Circuits device ZX60-V63+ which operates at a frequency range of \SI{0.05}{GHz} - \SI{6}{GHz}. It has a noise figure of $\approx$ \SI{3.6}{\decibel} at its lowest operating frequency of \SI{\approx 50}{MHz} \footnote{https://www.minicircuits.com/pdfs/ZX60-V63+.pdf}.\\

\textbf{SNAP Board} - The SNAP board used is specified by Xilinx Inc. \footnote{http://www.xilinx.com/products/silicon-devices/fpga/kintex-7.html}. The  SNAP board samples the signals it receives by 250 Msampl/s using internal ADCs where the signal of the clock is from the Valon 5007 frerquency synthesizer module. This creates a frequency range between \SI{0}{MHz} - \SI{125}{MHz} which contains 2048 channels. A SNAP board interacts with a raspberry pi and that is where the data is saved.\\

\subsection{Solar Power Supply System}
The power system of the ALBATROS-EGG uses generators to charge the batteries manually every once in a number of days. From the observation that has been made, the batteries take $\approx$ 2 days to discharge to a point where the system shuts down. Because of the weather conditions in Marion Island, an individual can sometimes be unable to go to the site to charge the batteries, which means that if the weather does not allow, the system can be shut down for a long period. Thus, a solar power system solution was introduced to the ALBATROS so that the system can continously run without having to be recharged manually and often.\\

The ALBATROS system is powered using \SI{24}{\volt} power which is stored and provided by two series-connected \SI{12}{\volt} deep-cycle lead-acid storage batteries. This power comes from the solar panel array consisting of nine flexible solar panels, each with a standard test condition capacity of \SI{110}{\watt}. The solar panel type is SunPower SPE-E-Flex-110. The panels are grouped in three parallel connected strings of three panels. Each group of three is mounted on its own structure. The panels include diodes to bypass shaded or defective cells, and also to prevent backwards current flow in the event an entire string of three panels is not illuminated while the others are producing power. \\

The Victron BlueSolar MPPT 50$\vert$35 charge controller outputs a data packet every second, to an Arduino based data logger, which controls a switch that determines when the rest of the system runs. It also converts the solar panel voltage to \SI{24}{\volt} to control the battery charging. The observed parameters are then stored to the raspberry pi.

\section{Pathfinder Installations and Preliminary Observations}
\subsection{Marion Island}
 Marion Island is a research location managed by the South African National Antarctic Programme (SANAP) and is located at \ang{46;54;45}S, \ang{37;44;37}E on the sub-antarctic of the Southern Indian Ocean. It is $\approx$\SI{2000}{\kilo\metre} from the nearest continental landmass which is also an approximate maximum distance for meteor scattering. The island has an area of \SI{290}{\kilo\metre\squared} and is only accessible once in April of every year by means of the S. A. Agulhas vessel owned by the Department of Environmental Affairs.  
 
 \subsection{Experiments}
 The ALBATROS-EGG shown in \autoref{Fig:ALBATROS-EGG} was installed on the PRI\textsuperscript{z}M \cite{2019JAI.....850004P} site which is located at \ang{46;53;13}S, \ang{37;49;10.7}E and the first ALBATROS station shown in \autoref{Fig:autonomous} is located \ang{46;52.205;}S, \ang{37;50.612;}E.\\
 

	\begin{figure}[h]
		\begin{center}
			\includegraphics[width=0.7\linewidth]{Figures/autonomous.jpg}
			\caption{}
			\label{Fig:autonomous}
		\end{center}
	\end{figure}

 \subsection{Preliminary Observations}
	
\autoref{Fig:auto} and \autoref{Fig:fringes} shows the initial results from the two interferometric array. These results are an encouraging factor to proceed with the development of the autonomous stations.\autoref{Fig:auto} shows the raw ALBATROS-EGG autospectra which where the waterfall plot was taken from one polarization (pol0) over an interval of 3 days. The Galaxy rising/setting is clearly visible in the structure. There are also ripples in frequency because of uncalibrated data, and the ripples arise from reflections in the cables. There is a qualitative difference between daytime and nighttime data and this shows that the contamination from shortwave radio drops off significantly at night, when the ionosphere becomes quieter.

\begin{figure}[h!]
	\begin{center}
		\includegraphics[width=0.8\linewidth]{Figures/Raw-ALBATROS-autospectra.PNG}
		\caption{Raw ALBATROS-EGG Autospectra}
		\label{Fig:auto}
	\end{center}
\end{figure}

\autoref{Fig:fringes} shows the first fringes that were detected by the the ALBATROS-EGG. It is dinstictly visible from \autoref{Fig:fringes} that fringes show recurrent structure down to a frequency of as low as 10 MHz without data processing or data cuts.

\begin{figure}[ht!]
	\begin{center}
		\includegraphics[width=0.7\linewidth]{Figures/First-fringes-of-ALBATROS-EGG.PNG}
		\caption{First Fringes from ALBATROS-EGG}
		\label{Fig:fringes}
	\end{center}
\end{figure}


		
\section{Future Work}

Marion Island huts which are potential stations for the ALBATROS have a ring-like pattern which is appropriate for imaging and produces a FWHM synthesized beam of 8' at 5 MHz as shown in \autoref{Fig:10}. This beam is a notable advancement over existing measurements to date.The ALBATROS main goal thus far will be to attempt to map high resolutions at low frequencies which is crucial before moving to measuring the dark ages.


\begin{figure}[h]
	\centering
	\begin{subfigure}[t]{0.5\textwidth}
		\centering
		\includegraphics[width=.9\linewidth]{Figures/site.PNG}
		\caption{Marion Island huts which are potential stations for the ALBATROS. The ring-like pattern is appropriate for imaging and produces a FWHM synthesized beam of 8' at 5 MHz.}
		\label{Fig:Marion}
	\end{subfigure}%
	~~~			
	\begin{subfigure}[t]{0.5\textwidth}
		\centering
		\includegraphics[width=.9\linewidth]{Figures/beam.png}
		\caption{Omnidirectional, dome shaped beam pattern which promises an improved resolution over the existing experiments to date.}
		\label{Fig:beam}
	\end{subfigure}
	\caption{Potential stations and a beam encouraging enough to implement ALBATROS experiment}
	\label{Fig:10}
\end{figure}


	
\section*{Acknowledgments}

	
\begin{thebibliography}{9}

\bibitem[Alexander et al.(1975)]{1975A&A....40..365A} Alexander, J.~K., Kaiser, M.~L., Novaco, J.~C., et al.\ 1975, {\it aap\/}, {\bf 40}, 365

\bibitem[Chen {et al.}(2019)]{2019arXiv190710853C} Chen, X., Burns, J., Koopmans, L., {\it et al.} [2019], arXiv e-prints, arXiv:1907.10853

\bibitem[Eastwood et al.(2018)]{2018AJ....156...32E} Eastwood, M.~W., Anderson, M.~M., Monroe, R.~M., et al.\ 2018, {\it aj\/}, {\bf 156}, 32

\bibitem[Ellingson and Kramer (2004)]{Memo28} Ellingson, S.~W., {Kramer}, W.~T. \ 2004, {\it Long Wavelength Array Memo (28)}

\bibitem[George {et al.}(2018)] {article} George, M., Orchiston, W., Wielebinsk, R., {\it et al.} [2018], Journal of Astronomical History and Heritage, {\bf 21}, 37

\bibitem[Hicks et al.(2012)]{2012PASP..124.1090H} Hicks, B.~C., Paravastu-Dalal, N., Stewart, K.~P., et al.\ 2012, {\it pasp\/}, {\bf 124}, 1090

\bibitem[Koopmans {et al.}(2019)]{2019arXiv190804296K} Koopmans, L., Barkana, R., Bentum, M., {\it et al.} [2019], arXiv e-prints, arXiv:1908.04296

\bibitem[Liu {et al.}(2013)]{2013PhRvD..87d3002L} Liu, A., Pritchard, J.~R., Tegmark, M., {\it et al.} [2013], {\bf 87}, 043002

\bibitem[Philip {et al.}(2019)]{2019JAI.....850004P} Philip, L., Abdurashidova, Z., Chiang, H.~C., {\it et al.} [2019], Journal of Astronomical Instrumentation, {\bf 8}, 19500

\bibitem[Pober {et al.}(2014)]{2014ApJ...782...66P} Pober, J.~C., Liu, A., Dillon, J.~S., {\it et al.} [2014], {\bf 782}, 66

\bibitem[Ray et al.(2006)]{Memo35} Ray, P.~S., Ellingson, S.~W., Fisher R., {\it et al.} [2006] {\it Long Wavelength Array Memo (35)}

\bibitem[Roger et al.(1999)]{1999A&AS..137....7R} Roger, R.~S., Costain, C.~H., Landecker, T.~L., et al.\ 1999, {\it aaps\/}, {\bf 137}, 7

\bibitem[Weiler {et al.}(1988)]{1988A&A...195..372W} Weiler, K.~W., Johnston, K.~J., Simon, R.~S., {\it et al.} [1988], {\it aap\/}, {\bf 195}, 372






\end{thebibliography}
\end{document}
