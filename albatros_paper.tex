\documentclass[12pt]{article}
\usepackage{siunitx}
\usepackage{amsmath}
\usepackage{amsfonts}
\usepackage{amssymb}
\usepackage[bottom]{footmisc}
%\usepackage{palatino}
%\usepackage{wrapfig}
\usepackage{graphicx}
\usepackage{url}
\usepackage[utf8]{inputenc}
\usepackage[english]{babel}
%\usepackage[left=3cm,right=3cm,top=3cm,bottom=3cm]{geometry}
%\usepackage[small,compact]{titlesec}
\usepackage{setspace}
\usepackage{hyperref,cleveref}
\usepackage{cite}
\usepackage{color}
%\usepackage{tikz}
\usepackage{caption}
\usepackage{subcaption}
\usepackage{siunitx}
%\usepackage{natbib}
\usepackage[authoryear]{natbib}
\usepackage{gensymb}
%\usepackage{lipsum}
\topmargin 0.0cm
\oddsidemargin 0.2cm
\textwidth 16cm 
\textheight 21cm
\footskip 1.0cm

\newenvironment{sciabstract}{%
\begin{quote} }
{\end{quote}}

\title{Title} 

\author
{Author 1,$^{1\ast}$ Author 2,$^{1}$ Author 3$^{2}$\\
\\
\normalsize{$^{1}$Department, University and Address 1}\\
\normalsize{$^{2}$Department, University and Address 2}\\
}

\begin{document} 

% Double-space the manuscript.

\baselineskip24pt

% Make the title.

\maketitle 

\begin{sciabstract}
    The low frequency radio astronomy has the highest potential in discovering the history of the Universe, this includes observations of the first stars and the mapping of dark ages. The Array of Long Baseline Antennas for Taking Radio Observations from the Sub-Antarctic (ALBATROS) is a new interferometric array. It consists of autonomous antenna stations that will map the low-frequency sky from Marion Island. One autonomous station was deployed in Marion Island in April 2019. The operating frequency range is 1.2–81 MHz with baselines of $\approx$20 km. A two element inteferometer, the ALBATROS - Exploratory Gizmo on the Ground (ALBATROS-EGG) was deployed in Marion Island in April 2018. So far, the inteferometer is functional and is detecting different sources. \\
\end{sciabstract}


\section*{Introduction}
The 21 cm wavelength of hydrogen gas is being observed by several experiments which are modelled for the purpose of Hydrogen mapping in our Universe. This hydrogen line is a significant mechanism as it helps in the probing of the dark ages to the epoch of reionization (EoR)  \citep{2013PhRvD..87d3002L,2014ApJ...782...66P}.

Comprehensive reviews of experimental efforts exist elsewhere but none of them have made measurements at the lowest frequencies of $\lessapprox$ 30 MHz. This is due to the challenges namely, the ionospheric effects, radio frequency intereference (RFI), Galactic emission and instrumental systematics \citep{2018arXiv180609531P}. Two of these experiments represent the lowest frequencies measured to date (Reber’s antenna, RAE-B), and the other two represent the highest resolutions achieved in this frequency range (DRAO, OVRO-LWA).\\

Measurements of the radio sky at $\approx$ 100 MHz and below have the capability of unlocking the new observational window in the history of the universe. At the lowest frequencies (tens of MHz), subsequent observations may permit us to probe the cosmic "dark ages," one day, which is an epoch that is obscure to date \citep{2019arXiv190710853C}. The state of the art among ground-based measurements dates from the 1960s, when Grote Reber caught brief glimpses of the $\approx$ 2 MHz sky at low resolution. This paper will describe a new project that aims to map the low-frequency sky from Marion island using an array of autonomous antenna stations. The final array will consist of  $\approx$ 10 antennas operating at 1.2-81 MHz with baselines up to 20 km. A two-element pathfinder was deployed in April 2018, the first autonomous station was deployed in April 2019 and there'll be discussion of the preliminary observations and upcoming hardware development plans.






\section*{Overview of the Instrument}
	\citep{2019arXiv190804296K}

\section*{Pathfinder Installations and Preliminary Observations}


\section*{Future Work}



\newpage		
\bibliography{ALBATROS_Paper}		
\bibliographystyle{apalike}

\end{document}




















