%%%%%%%%%%%%%%%%%%%%%%%%%%%%%%%%%%%%%%%%%%%%%%%%%%%%%%%%%%%%%%%%%%%%%%%%%%%%
%% Trim Size : 11in x 8.5in
%% Text Area : 9.6in (include Runningheads) x 7in
%% ws-jai.tex, 26 April 2012
%% Tex file to use with ws-jai.cls written in Latex2E.
%% The content, structure, format and layout of this style file is the
%% property of World Scientific Publishing Co. Pte. Ltd.
%%%%%%%%%%%%%%%%%%%%%%%%%%%%%%%%%%%%%%%%%%%%%%%%%%%%%%%%%%%%%%%%%%%%%%%%%%%%
%%

%\documentclass[draft]{ws-jai}
\documentclass{ws-jai}
\usepackage[flushleft]{threeparttable}
\usepackage{siunitx}
\usepackage{amsmath}
\usepackage{gensymb}
\usepackage[colorlinks=true,allcolors=blue]{hyperref}
\usepackage{graphicx}
\usepackage{caption}
\usepackage{subcaption}
\usepackage[bottom]{footmisc}
\usepackage{url}
\usepackage{natbib}


% Define stuff here
\def\albatros{ALBATROS}
\def\prizm{PRI$^{\rm Z}$M}
\newcommand{\attention}[1]{\textcolor{red}{\bf {#1}}}

\newcommand{\aap}{Astronomy and Astrophysics}
\newcommand{\aaps}{Astronomy and Astrophysics Supplements}
\newcommand{\aj}{Astronomical Journal}
\newcommand{\apj}{Astrophysical Journal}
\newcommand{\grl}{Geophysical Research Letters}
\newcommand{\jgr}{Journal of Geophysical Research}
\newcommand{\mnras}{Monthly Notices of the Royal Astronomical Society}
\newcommand{\nar}{New Astronomy Reviews}
\newcommand{\pasa}{Publications of the Astronomical Society of Australia}
\newcommand{\pasp}{Proceedings of the Astronomical Society of the Pacific}
\newcommand{\prd}{Physical Review D}
\newcommand{\prl}{Physical Review Letters}

\begin{document}

\catchline{}{}{}{}{} % Publisher's Area please ignore

\markboth{Author's Name}{ALBATROS instrument}

\title{The Array of Long-Baseline Antennas for Taking Radio
  Observations from the Sub-Antarctic}

\author{First Author$^{2}$, Second Author$^{3}$, Third Author$^{3}$ and Fourth Author$^{4}$}

\address{
$^{2}$Department, University Name, City, State ZIP/Zone, Country, fauthor@university.com\\
$^{3}$Group, Company, Address, City, State ZIP/Zone, Country\\
$^{4}$Group, Company, Address, City, State ZIP/Zone, Country, fauthor@company.com
}

\maketitle

\corres{$^{2}$Corresponding author.}

\begin{history}
\received{(to be inserted by publisher)};
\revised{(to be inserted by publisher)};
\accepted{(to be inserted by publisher)};
\end{history}

\begin{abstract}
Measurements of redshifted 21-cm emission of neutral hydrogen at
$\lesssim30$~MHz have the potential to probe the cosmic ``dark ages,''
a period of the universe's history that remains unobserved to date.
Observations at these frequencies are exceptionally challenging
because of bright Galactic foregrounds, ionospheric contamination, and
terrestrial radio-frequency interference.  Very few sky maps exist at
$\lesssim30$~MHz, and most have modest resolution.  We introduce the
Array of Long Baseline Antennas for Taking Radio Observations from the
Sub-Antarctic (\albatros), a new experiment that aims to image
low-frequency Galactic emission with an order-of-magnitude improvement
in resolution over existing data.  The \albatros\ array will consist
of antenna stations that operate autonomously, each recording baseband
data that will be interferometrically combined offline.  The array
will be installed on Marion Island and will ultimately comprise 10
stations, with an operating frequency range of 1.2--125~MHz and
maximum baseline lengths of $\sim20$~km.  We present the
\albatros\ instrument design and discuss pathfinder observations that
were taken from Marion Island during 2018--2019.
\end{abstract}

\keywords{cosmology: observations; dark ages; instrumentation: interferometers}

\attention{Open task for any/all: fix references}

\section{Introduction}

Measurements of redshifted \SI{21}{\cm} emission of neutral hydrogen
across a wide range of radio frequencies have the potential to
elucidate the universe's history from the cosmic ``dark ages'' up to
the formation of large-scale
structures~\citep{2020PASA...37....2W,2013PhRvD..87d3002L,2014ApJ...782...66P}.
The dark ages, which occurred after recombination and correspond to a
period when the universe was filled with neutral hydrogen, are
unexplored to date and represent one of the final observational
frontiers in cosmology.  The number of independent modes that are
contained in the matter power spectrum in this epoch is orders of
magnitude greater than the corresponding number for cosmic microwave
background (CMB) measurements~\citep{2004PhRvL..92u1301L}.  Redshifted
21-cm measurements of the dark ages therefore hold the potential to
significantly improve cosmological parameter constraints over those
currently derived from the CMB; however, the required observational
frequencies of $\lesssim 30$~MHz are exceptionally difficult to
access.  The primary experimental challenges include Galactic
foreground emission, ionospheric interference, and terrestrial
radio-frequency interference (RFI).

Very few experiments have surveyed the radio sky at $\lesssim 30$~MHz,
and here, we highlight only four specific examples.  At the very
lowest frequencies, the state of the art among ground-based
measurements dates from the 1950s, when Reber and Ellis caught brief
glimpses of the $2.1$~MHz sky at $\sim 5$\degree\ resolution, using an
array of 192~dipoles~\citep{1956JGR....61....1R}.  A few space-based
missions have also performed measurements at similarly low frequency
ranges; for example, the Radio Astronomy Explorer-2 operated at
\SIrange{0.025}{13}{\MHz} with $\sim 10$\degree\ resolution at
\SI{4.7}{\MHz}~\citep{1975A&A....40..365A}.  Measurements with
resolution finer than few-degree scales exist mainly at higher
frequencies.  For example, the Dominion Radio Astrophysical
Observatory surveyed the northern sky at \SI{22}{\MHz} with
\SIrange{1.1}{1.7}{\degree} resolution~\citep{1999A&AS..137....7R},
and most recently, the Owens Valley Long Wavelength Array mapped the
sky with \SI{15}{\arcminute} resolution between 36.528~MHz and
73.152~MHz~\citep{2018AJ....156...32E}.  Although this experimental
list is not comprehensive, it does illustrate the dearth of
information about the $\lesssim 30$~MHz sky and the lack of high
resolution measurements at the lowest frequencies.  The first step
toward laying the groundwork for possible measurements of the dark
ages is obtaining an improved map of Galactic foregrounds at $\lesssim
30$~MHz.  In addition to serving as a stepping stone for future
cosmological constraints from the dark ages, maps at these frequencies
can also shed new light on Galactic astrophysics.  Above $\sim30$~MHz,
synchrotron emission from the Galaxy follows a power law with a
frequency dependence of $\nu^{-0.7}$.  At lower frequencies,
synchrotron self-absorption becomes non-negligible, and the spectrum
transitions to $\nu^{+5/2}$ dependence.  Since self-absorption causes
the optical depth along the line of sight to depend on frequency,
observations at $\lesssim 30$~MHz have the potential to probe the
three-dimensional cosmic ray structure of the
Galaxy~\citep{2002ApJ...575..217P}.  Low-frequency observations can
also be used to image radio recombination lines (RRLs), which provide
a means for studying cool, low-density regions of the interstellar
medium (ISM).  These absorption lines arise from Rydberg atoms and are
highly sensitive to the surrounding environment.  The RRL spectrum and
absorption profiles can therefore constrain the detailed properties of
the ISM regions in which the Rydberg atoms are
formed~\citep{2009NewAR..53..259G, 2007MNRAS.374..852S}.  Finally,
observations at $\lesssim30$~MHz may provide new views of Earth's
ionosphere, which absorbs and refracts at radio frequencies and
becomes completely opaque below the plasma cutoff frequency.  The
cutoff frequency and the levels of absorption and refraction are
time-varying and spatially dependent.  Experiments that image the
radio sky at multiple frequencies bracketing the plasma cutoff can
therefore simultaneously image the ionosphere and provide spectral
riometry data for space weather studies~\citep{2014GeoRL..41.5370K}.

\begin{figure}
  \begin{center}
    \includegraphics[width=0.6\linewidth]{Figures/marion_domec_hobart.png}
    \caption{Minimum plasma frequency predicted by the International
      Reference Ionosphere model during the last solar minimum.  At
      some locations on Earth, the plasma frequency may drop as low as
      $\sim1.5$~MHz.}
    \label{Fig:iri}
  \end{center}
\end{figure}

There are proposed efforts to perform new low-frequency measurements
from space, where there is no contamination from the ionosphere, and
the lunar farside can potentially block RFI from the
Earth~\citep{2019arXiv190710853C, 2019arXiv190804296K}.  Although the
combination of ionosphere and RFI significantly impedes low-frequency
radio observations from most locations on Earth, preliminary
observations from Marion Island~\citep{2019JAI.....850004P} suggest
that such observations may still be accessible from carefully selected
locations and with new technology developments.  In this paper, we
present the Array of Long Baseline Antennas for Taking Radio
Observations from the Sub-Antarctic (\albatros), a new experimental
effort that aims to map the low-frequency sky using an array of
autonomous antenna stations.  We describe the overall instrument
design and preliminary measurements from engineering runs that were
performed on Marion Island during 2018--2019.

\begin{figure}
    \centering
    \begin{subfigure}[t]{0.6\textwidth}
        \centering
        \includegraphics[width=\linewidth]{Figures/marion_map/marion_map_annotated.jpg} 
        \caption{} \label{Fig:marion_map}
    \end{subfigure}
    \hfill
    \begin{subfigure}[t]{0.39\textwidth}
      \centering
        \includegraphics[width=\linewidth]{Figures/marion_beam_huts_2020.jpg}
        \caption{} \label{Fig:marion_beam}
    \end{subfigure}
    \caption{{\bf (a)} Map of Marion Island.  The
      \albatros\ pathfinder antennas are currently installed at the
      \prizm\ site and at the hydro shack (gray markers).  The black
      markers denote the eight coastal huts, which will be used for
      future \albatros\ antenna installations.  The white markers
      denote other available infrastructure points that will not be
      used for antennas.  {\bf (b)} Synthesized beam at 5~MHz from the
      full \albatros\ array, using the existing and planned
      installation locations shown on the map.  Using an octave of
      bandwidth spanning 3.5--7~MHz, we obtain a synthesized beam with
      a full width of $\sim7'$.}\label{Fig:marion_map_beam}
\end{figure}

\section{ALBATROS overview}

% Write text here about the overall final configuration.  Describe the
% site, show synthesized beam, discuss high level system requirements
% (power, data storage, etc).  Include deployment schedule and introduce
% 2-element pathfinder in the following section.

The primary requirements that drive the design for a low-frequency
imaging experiment are 1) desired resolution, 2) low RFI, and 3) quiet
ionospheric conditions.  As a benchmark, an interferometer operating
at 30~MHz requires order-of-magnitude baseline lengths of $\sim 1$~km
to achieve a resolution of $\sim 1$\degree.  This length scales
inversely with frequency, and therefore $\sim 10$-km lengths are
required at few-MHz frequencies to improve upon the resolutions
achieved to date.  The experiment installation site must be remote to
keep RFI to a minimum, and polar or near-polar latitiudes generally
have lower ionospheric plasma frequency cutoffs relative to other
locations on Earth.  \autoref{Fig:iri} illustrates predictions from
the International Reference Ionosphere
model~\citep{2018AdRS...16....1B} for the minimum plasma frequency
during the last solar minimum, which started in approximately 2007.
Three locations are shown: Marion Island, the focus of this work;
Dome~C in Antarctica, which is another isolated location used for
astronomical observations; and Hobart, Tasmania, where Reber performed
his $\sim2$~MHz observations.  The model predictions illustrate that
the ionospheric plasma cutoff frequency at Marion Island may drop as
low as $\sim1.5$~MHz during solar minima.  Since we are currently
experiencing another solar minimum~\citep{2018NatCo...9.5209B}, the
timing is opportune for new low-frequency observations.

Marion Island is a research base that is located in the southern
Indian Ocean at \ang{46;54;45}S, \ang{37;44;37}E and is operated by
the South African National Antarctic Programme.  The island lies
roughly \SI{2000}{\kilo\metre} from the nearest continental landmasses
and has an exceptionally quiet RFI
environment~\citep{2019JAI.....850004P}.  As illustrated in
\autoref{Fig:marion_map}, Marion has an area of 335~km$^2$ and can
therefore support antenna installations with $>10$-km baseline
lengths.  The main Marion base is located on the northeast side of the
island, and there are eight rest huts along the coast (Cape Davis,
Grey-headed, Kildalkey, Mixed Pickle, Repettos, Rooks, Swartkops,
Watertunnel) and one in the interior (Katedraal) that can serve as
existing infrastructure points for antenna installations.  The planned
\albatros\ installation sites include the coastal huts, but exclude
the main base and Katedraal for RFI and accessibility reasons,
respectively.  \autoref{Fig:marion_map} also shows the locations of
the \albatros\ pathfinder antennas that are currently installed at the
\prizm\ site and at the hydro shack.

Using the eight coastal huts, the \prizm\ site, and the hydro shack as
the nominal \albatros\ installation locations, the computed
synthesized beam is as shown in \autoref{Fig:marion_beam}.  The beam
width at 5~MHz is roughly \SI{7}{\arcminute}, which represents over an
order of magnitude improvement in resolution over other existing
measurements.  One of the challenges in constructing an interferometer
array on Marion Island is that the rugged terrain precludes the
possibility of directly cabling and correlating antennas across large
distances.  The final \albatros\ antenna stations will therefore
operate {\it autonomously}, recording baseband data over extended
periods of time for subsequent offline correlation.  We have conducted
two engineering runs: 1) a two-element, directly correlated pathfinder
to qualitatively understand the sky signal, and 2) a single station to
test the readout and power handling technology that are required for
autonomous operation.

\section{Two-element pathfinder}

\begin{figure}
  \begin{center}
    \includegraphics[width=0.7\linewidth]{Figures/albatros_2elem/albatros_2elem.pdf}
    \caption{The two-element, directly correlated
      \albatros\ pathfinder installed at the \prizm\ site.  Two
      dual-polarization antennas are separated by roughly 110~m on an
      east--west baseline. Coaxial cables connect the antennas to a
      shipping container that houses the readout electronics and
      serves as the ``command module.''}
    \label{Fig:albatros2}
  \end{center}
\end{figure}

\begin{figure}
  \begin{center} \includegraphics[width=1.0\linewidth]{Figures/albatros_2elem_schematic/albatros_2elem_schematic.pdf}
    \caption{Two-element \albatros\ pathfinder block diagram.  Signals
      from two dual-polarzation LWA antennas are amplified by
      front-end active baluns~\citep{2012PASP..124.1090H}.  The
      antennas are connected via 100-m coaxial cables to the back-end
      readout electronics, which are housed in a Faraday cage denoted
      by the dashed box.  Each of the four antenna outputs is passed
      to a second-stage electronics chain consisting of filters and
      further amplfication.  The signals are digitized at 250~Msamp/s
      by a SNAP board, which includes an on-board FPGA that computes
      auto- and cross-spectra from and between the four inputs.  A
      Raspberry Pi controls the SNAP board and saves the data.}
    \label{Fig:albatros2_schem}
  \end{center}
\end{figure}

The first exploratory \albatros\ measurements were conducted with a
two-element pathfinder that used direct correlation (without
autonomous operation).  \autoref{Fig:albatros2} illustrates this
pathfinder, which was installed at the \prizm\ site (\ang{46;53;13}S,
\ang{37;49;10.7}E) in April 2018.  The system block diagram is shown
in \autoref{Fig:albatros2_schem}, and each of the subsystems is
described in detail below.

\subsection{Antenna}\label{s:antenna}

We employ two dual-polarization Long Wavelength Array (LWA) dipole
antennas~\citep{Memo28}.  The LWA antennas have a long development
history, are well characterized, and are simple to install and
physically robust.  The antennas form an east--west baseline with a
separation of \SI{110}{m}, and the polarizations are aligned with the
cardinal directions.  Welded wire mesh screens, roughly 3~m on a side,
are installed on the ground below the antennas.
% do we want/need to say anything about the omnidirectional primary beam?

\subsection{Front-end active balun}\label{s:fee}

The LWA active-balun front-end electronics (FEE) circuit uses a
MiniCircuits GALI-74+ MMIC to amplify each dipole leg against ground,
presenting each leg with a 50$\ohm$ impedance and providing a nominal
gain of 25~dB. The two GALI-74+ outputs are differenced using a
passive {180\degree} hybrid coupler. The coupler output is filtered by
a $\sim$150~MHz lowpass and receives an additional 12~dB of gain from
a MiniCircuits GALI-6+ MMIC. This last amplifier drives the output
signal onto a 100-m 50$\ohm$ coaxial cable. Thus, neglecting mismatch
loss between the dipoles (electrically small in the frequency range of
interest) and the 100{\ohm} input impedance, as well as the hybrid
insertion loss ($<$1dB), the front-end electronics provide a gain of
$\sim$37~dB~\citep{2012PASP..124.1090H}.  Each FEE circuit is powered
by 16~V, which is fed on the coaxial cable through a bias tee.

% EE citing the noise figure of the GALI-74+ is particulary misleading because of the large mismatch loss, which directly scales up the noise figure in the HF band. 

%% It seems to me that this is too much detail to go into in describing an already published design.
% All the front end components were incorporated for in a double sided
% printed circuit board (PCB) as shown in \autoref{Fig:Balun} and the
% block diagram is shown in \autoref{Fig:Balun Schematic}. The
% Monolithic Microwave Integrated Circuits (MMICs) is the used design
% for the PCB. One side of the PCB is populated with components and
% the other side is a solid copper ground plane aperiodically stitched
% to the grounded copper on the side populated with components. The
% receiver system is made up of the active balun, filter and the gain
% stage that connects to the \SI{100}{\metre} coaxial cable which is
% connected to the back end \cite{2012PASP..124.1090H}.

% The input impedance (Z\textsubscript{o}) of \SI{50}{\ohm} is
% introduced to the dipole by the active balun. The signal is then fed
% through an amplifier which amplifies it by \SI{+24}{\decibel} of
% gain. The balanced signal is then converted to unbalanced through a
% 180 \degree hybrid coupler. The band pass filter (BPF) receives the
% single ended signal in order for it to reject all the frequencies
% which are not within the range of interest. The signal gets fed to a
% second amplifier which again amplies it by \SI{+24}{\decibel} of gain
% and the output impedance of the FEE is matched to a \SI{50}{\ohm}
% coaxial cable. The bias tee is responsible for providing power to the
% FEE and extracts the RF signal by the use of the coaxial cable. This
% unit has an overall gain of $\approx$ \SI{35}{\decibel} and an overall
% noise figure of $\approx$ \SI{2.7}{\decibel} to $\approx$
% \SI{2.9}{\decibel} \cite{Memo35}.

\subsection{Back-end electronics}

The back-end readout electronics are housed in a Faraday cage located
100~m away from the antennas to mitigate possible self-generated RFI.
Each of the four antenna signals is passed to a second-stage
electronics chain consisting of an amplifier (Minicircuits ZX60-V63+),
and a pair of high- and low-pass filters (Minicircuits ZFHP-1R2+ and
SLP-90+) that together band-limit the signal to 1.2--\SI{81}{MHz}.
The amplifier has a nominal gain of 20~dB, and the high- and low-pass
filters contribute nominal insertion losses of 0.2~dB and 0.14~dB,
respectively.

A Smart Network ADC Processor~\citep[SNAP;][]{2016JAI.....541001H}
board digitizes the RF signals at 250~Msamp/s and uses a Xilinx
Kintex~7\footnote{http://www.xilinx.com/products/silicon-devices/fpga/kintex-7.html}
FPGA to calculate full cross-correlations of the four inputs,
producing four auto- and six cross-spectra as outputs, over 2048
channels spanning the frequency range 0--125~MHz.  \attention{details
  about channelization here} A Valon 5007 frequency synthesizer module
provides the clock signal for the SNAP board.  A Raspberry Pi~3B+
(RPi) single board computer controls the SNAP board and receives the
spectra via GPIO connections, and the spectra are saved to an on-board
SD card.  An Adafruit Ultimate GPS
module\footnote{\url{https://www.adafruit.com/product/746}}, connected
to an active external GPS antenna, provides absolute timing for the
RPi.

\subsection{Power}

A bank of four 12-V, 200-Ah AGM batteries, connected in parallel,
powers the two-element pathfinder system.  The batteries are manually
charged with a Honda EU30is generator that is housed on site.  The
total power draw of the two-element pathfinder is $\sim45$~W, and when
fully charged, the battery bank can power the system for about a week.
The raw battery voltage is passed to several DC/DC converters that
supply regulated voltages to the SNAP board, FEE, amplifiers and the
clock.

\subsection{Software and data acquisition}

\attention{text goes here}

%%%%%%%%%%%%%%%%%%%%%%%%%%%%%%%%%%%%%%%%%%%%%%%%%%%%%%%%%%%%%5

\section{Single autonomous station pathfinder}

\begin{figure}
    \centering
    \begin{subfigure}[t]{0.48\textwidth}
        \centering
        \includegraphics[width=\linewidth]{Figures/autonomous.jpg} 
        \caption{} \label{Fig:autonomous_antenna}
    \end{subfigure}
    \hfill
    \begin{subfigure}[t]{0.48\textwidth}
      \centering
        \includegraphics[width=\linewidth]{Figures/container.jpg}
        \caption{} \label{Fig:autonomous_electronics}
    \end{subfigure}
    \caption{{\bf (a)} Single autonomous station pathfinder installed
      at the hydro shack site.  The system is powered by a bank of
      solar panels that are visible in the background. {\bf (b)} A
      weather-proof container, which sits near the solar panels,
      houses the batteries, solar charge controller, and readout
      electronics.}\label{Fig:autonomous}
\end{figure}

\begin{figure}
  \begin{center}
    \includegraphics[width=\linewidth]{Figures/albatros_single_schematic/albatros_single_schematic.pdf}
    \caption{Single-antenna autonomous station block diagram.  A
      dual-polarization LWA antenna, equipped with a front-end active
      balun, connects via 50-m coaxial cables to the back-end readout
      electronics, housed in a Faraday cage denoted by the dashed box.
      Each of the two antenna signals is passed to a second-stage
      electronics chain consisting of filters and further
      amplfication.  The signals are digitized at 250~Msamp/s by a
      SNAP board, which includes an on-board FPGA that computes
      channelized baseband and spectra from both inputs.  A Raspberry
      Pi controls the SNAP board and receives the baseband data and
      spectra.  The baseband data are saved to external hard drives.
      The system is powered by solar panels that charge a 24-V battery
      bank.}
    \label{Fig:albatros1_schem}
  \end{center}
\end{figure}

The full ALBATROS configuration will comprise an array of 10
autonomous antenna stations, each recording baseband over a tunable
10--20~MHz window within the full 0--125~MHz operating range.  The
baseband data will be collected periodically and subsequently
correlated offline.  The ALBATROS stations, located at the eight
coastal hut sites plus the hydro shack and \prizm\ site, will be
separated by baselines of $\sim20$~km as shown in
\autoref{Fig:marion_map_beam}. One fully autonomous ALBATROS station,
shown in \autoref{Fig:autonomous}, was deployed in April 2019 at the
hydro shack location (\ang{46;52.205;}S, \ang{37;50.612;}E) on Marion
Island as a first step in testing the technology needed to establish
the full array.  The system block diagram is shown in
\autoref{Fig:albatros1_schem}.  The antenna and front-end active balun
in the single autonomous station are identical to those used in the
two-element pathfinder (\S\ref{s:antenna} and \S\ref{s:fee}), and the
back-end electronics and power system are described in detail below.

% HCC: we probably don't need this figure
% \begin{figure}[h]
% 	\begin{center}
% 		\includegraphics[width=0.5\linewidth]{Figures/balun.jpg}
% 		\caption{Unenclosed FEE mounted on the ALBATROS-EGG antenna supporting structure.} 
% 		\label{Fig:Balun}
% 	\end{center}
% \end{figure}

\subsection{Back-end electronics}

The back-end readout electronics are housed in a Faraday cage located
50~m away from the antennas. The analog signal chain consists of a
pair of high- and low-pass filters (Minicircuits ZFHP-1R2+ and
SLP-150+) that together band-limit the signal to 1.2--140~MHz, and the
filters are followed by a MiniCircuits ZFL-500+ amplifier.  In
contrast to the two-element pathfinder, the low-pass cutoff is
increased from 81~MHz to 140~MHz to capture downlink signals at
137--138~MHz from the ORBCOMM satellite constellation.
\attention{words here about how we'll use orbcomm for backup timing
  information}

As with the two-element pathfinder, a SNAP board digitizes each of the
two RF signals at 250~Msamp/s. In the autonomous station
configuration, the SNAP board ADCs are locked to a 10-MHz reference
produced by a Trimble Thunderbolt~E GPS-disciplined clock module.  The
SNAP board FPGA computes two data products: 1) channelized baseband
data for each polarization over tunable frequency windows within the
0--125~MHz operating range, with the options of 1-, 2-, or 4-bit
compression, and 2) auto- and cross-spectra from the two polarizations
over the full 0--125~MHz span, accumulated over few-second intervals.
An RPi~3B+ controls the SNAP board and and receives the auto- and
cross-spectra via GPIO connections, and the spectra are saved to an
on-board SD card.  The baseband data are passed from the SNAP board to
the RPi via ethernet and written to external hard drives.  The
introduction of gigabit ethernet with the RPi~3B+ model has enabled
the high data throughput associated with writing baseband.  As a
benchmark, 1-bit baseband recording of two polarizations over 10~MHz
of bandwidth yields an approximate data rate of 5~MB/s, or 0.4~TB/day.

\subsection{Correlation}

While the basics of 1-bit correlation are well known in the limit where the signal level is much lower than the noise, ALBATROS may well be in the high-signal regime.  We present the basics of 1-bit correlation here, and show that even in the high-signal limit, 1-bit correlation remains viable.
The fundamental output of a 1-bit correlator for real data is $x_{ij} \equiv \left < \tilde{E}_i \tilde{E}_j \right >$.  $ \tilde{E}_i$ is the quantized version of the underlying electric field $E_i$ where $ \tilde{E}_i=1$ for $E_i>0$ and $ \tilde{E}_i=-1$ for $E_i<0$.  Note that complex data can be handled as the combination of real components.  This output is non-linear in the underlying true signal and noise levels, since the output saturates at unity for perfectly correlated electric fields.  To get at the true sky signals, we will need to undo this nonlinearity (the so-called Van Vleck corrections).  For a 2-level correlator, the expected output can be related to the true signals as follows (Van Vleck \& Middleton, e.g. D'Addario):
\begin{eqnarray}
\label{eqn:1bit_output}
\left < \sin(\frac{\pi}{2}x_{ij})\right > = \frac{V_{ij}}{\sqrt{V_{ij}+N_i}\sqrt{V_{ij}+N_j}}
\end{eqnarray}
where $V_{ij}$ is the true visibility, and $V_{ij}+N_{i,j}$ is the total noise power measured by antennas $i$ and $j$.  Inverting this relation to get the true signal requires knowing the noise powers, which is not possible from the 1-bit data themselves.  However, the SNAP board calculates these power levels on timescales of a few seconds, much faster than the power levels change, and so by combining the single-station power measurements with the cross-station cross correlation, we will be able to derive the true sky visibilities.  

\begin{figure}
  \begin{center}
    \includegraphics[width=0.7\linewidth]{Figures/corr_efficiency.png}
    \caption{Correlator efficiency $\epsilon_{corr}$ for a 1-bit correlator as a function of signal-to-noise ratio (SNR).  To get the expected SNR from a 1-bit correlator, multiply the SNR derived from the radiometer equation by $\epsilon_{corr}$.  For low SNR, the efficiency is $2/\pi \sim 0.64$, and it drops monotonically as the SNR increases.  The drop is relatively gentle, only changing by a factor of $\sim 2$ for SNR=2.  For baseline separations of many wavelengths, the correlated component of the electric field is likely to be a small fraction of the total power, so ALBATROS is unlikely to lose more than a factor of $\sim 2$ by using 1-bit correlation.}
    \label{fig:1bit_efficiency}
    \end{center}
\end{figure}

One might worry that at high signal levels, the saturation of the 1-bit cross correlation would lead to the noise exploding.  In practice the rolloff in sensitivity is relatively mild, especially in the case of long baselines.  It can be expressed in terms of the correlator efficiency $\epsilon_{corr}$, which is defined to be the ratio of the measured signal-to-noise ratio to the ideal (infinite precision) SNR.  As seen in Figure \ref{fig:1bit_efficiency}, $\epsilon_{corr}$ only drops by a factor of $\sim$2 as the signal power goes from 0 to a few times the noise power.  We stress that the signal level in question is the part that correlates between the two antennas.  In the case of long baselines (where long means the fringe spacing is small compared to the primary beam/antenna response pattern) that are not dominated by a single bright source, the signal power will always be smaller than the noise power, and so $\epsilon_{corr}$ is unlikely to drop below $\sim$0.5\footnote{A quick estimate can be made by analogy to the behavior of dishes in the UV plane in the flat sky approximation.  While all power in the UV plane (plus any system noise) goes into an individual antenna's electric field, the correlated part only has contribution from the area in the UV plane within the width of the UV-space primary beam of the UV-space coordinate.  If the sky can be described as a Gaussian random field, in the most pessimistic case this is roughly the aperture filling factor of the antenna pair.  In our case, where the antennas are approximately dipolar, the filling factor will be the square of the wavelength over the baseline length.  As long as the baselines are several wavelengths or longer, the correlated power will be small, and the correlator efficiency will be close to the ideal 1-bit value of $\frac{2}{\pi}$.}.


\subsection{Solar power system}
\attention{Nivek, Tankiso, Eamon?} \\
Although the first-deployed two-element pathfinder is located at a reasonable walking distance from the main base and uses a generator to periodically recharge batteries, the bulk of the antenna stations that will comprise the full array will be located at points farther removed from the main base, and will require a fully autonomous energy source. So in deploying the autonomous station pathfinder, we developed a solar charging system to power the station.

The autonomous station is powered by two series-connected 12V deep-cycle lead-acid batteries, charged by an array of nine  SunPower SPE-E-Flex-110 solar panels. These panels each have a standard test capacity of 110W. Although only $\sim$50W is required to run the station (and the station may typically be powered on only over night) the $\sim$1kW charging capacity is sized to allow the charge level to recover quickly on a short sunny day in winter if it runs down due to several consecutive overcast days. 

The nine solar panels are distributed between three custom-designed structures built from aluminum extrusion, with rigid metalized plastic panels backing the semi-flexible solar cells. The structures are oriented due north, and are designed to incline the solar panels at a relatively steep angle so as to maximize peformance under winter conditions, when sunlight hours will be at their minimum. The solar panel mounts have been designed to withstand frequent gale-force winds on Marion. While the mounting structures themselves appear to be strong enough, the greatest challenge in deployment has been to find an adequate anchoring technique in the volcanic ground.

Each group of three panels is wired in series, and the three series strings are connected in parallel to a Victron BlueSolar MPPT 50$\vert$35 charge controller, which optimizes power transfer from the solar array when charging is required, and also monitors charge level, reducing charge current when the battery bank is fully charged.

A power logging and control system runs on an Arduino, which logs information received from the Victron charge controller to an SD card, and switches power on and off to the receiver chassis. Logged power data will be used to refine powers system requirements for future autonomous station deployments. The on/off control is necessary to prevent battery system  damage from overly deep discharge. The system can also be configured to conserve battery power by running on a schedule between particular hours, typically overnight, since visibility may be better and ionosopheric noise levels lower.

The power logging and control system, along with the Victron charge controller and an EMI filter, are housed together in an aluminum box. The EMI filter reduces conducted emissions on the photovoltaic side of the solar charge controller, which could radiate from the solar array and connecting wires. (It may be noted however, that signals collected at night, which are expected to be more valuable, are not at risk of solar charge controller EMI contamination.)

\iffalse
The ALBATROS system is powered using \SI{24}{\volt} power which is stored and provided by two series-connected \SI{12}{\volt} deep-cycle lead-acid storage batteries. This power comes from the solar panel array consisting of nine flexible solar panels, each with a standard test condition capacity of \SI{110}{\watt}. The solar panel type is SunPower SPE-E-Flex-110. The panels are grouped in three parallel connected strings of three panels. Each group of three is mounted on its own structure. The panels include diodes to bypass shaded or defective cells, and also to prevent backwards current flow in the event an entire string of three panels is not illuminated while the others are producing power. \\

The Victron BlueSolar MPPT 50$\vert$35 charge controller outputs a data packet every second, to an Arduino based data logger, which controls a switch that determines when the rest of the system runs. It also converts the solar panel voltage to \SI{24}{\volt} to control the battery charging. The observed parameters are then stored to the raspberry pi.
\fi

\section{Preliminary observations}

\attention{NEED UPDATED AND NEW PLOTS, NEED VOLUNTEERS FOR THIS:
  \begin{itemize}
    \item{Improved waterfall plots from 2-element pathfinder}
    \item{Plot showing that the solar power works (sort of)}
    \item{Some sort of...data...plot for the autonomous station.  Can
      we say anything about baseband performance?}
  \end{itemize}
}
 
\autoref{Fig:auto} and \autoref{Fig:fringes} shows the initial results from the two interferometric array. These results are an encouraging factor to proceed with the development of the autonomous stations.\autoref{Fig:auto} shows the raw ALBATROS-EGG autospectra which where the waterfall plot was taken from one polarization (pol0) over an interval of 3 days. The Galaxy rising/setting is clearly visible in the structure. There are also ripples in frequency because of uncalibrated data, and the ripples arise from reflections in the cables. There is a qualitative difference between daytime and nighttime data and this shows that the contamination from shortwave radio drops off significantly at night, when the ionosphere becomes quieter.

\begin{figure}[h!]
	\begin{center}
		\includegraphics[width=0.8\linewidth]{Figures/Raw-ALBATROS-autospectra.PNG}
		\caption{Raw ALBATROS-EGG Autospectra}
		\label{Fig:auto}
	\end{center}
\end{figure}

\autoref{Fig:fringes} shows the first fringes that were detected by the the ALBATROS-EGG. It is dinstictly visible from \autoref{Fig:fringes} that fringes show recurrent structure down to a frequency of as low as 10 MHz without data processing or data cuts.

\begin{figure}[ht!]
	\begin{center}
		\includegraphics[width=0.7\linewidth]{Figures/First-fringes-of-ALBATROS-EGG.PNG}
		\caption{First Fringes from ALBATROS-EGG}
		\label{Fig:fringes}
	\end{center}
\end{figure}

% \section{Future Work}
	
\section*{Acknowledgments}

We are extremely grateful for the anemic carrots on Marion.  Yum.

\bibliographystyle{apj}                                                                                                                                                                                                   
\bibliography{albatros_paper}{}


	
%\begin{thebibliography}{9}
%
%\bibitem[Alexander et al.(1975)]{1975A&A....40..365A} Alexander, J.~K., Kaiser, M.~L., Novaco, J.~C., et al.\ 1975, {\it aap\/}, {\bf 40}, 365
%
%\bibitem[Chen {et al.}(2019)]{2019arXiv190710853C} Chen, X., Burns, J., Koopmans, L., {\it et al.} [2019], arXiv e-prints, arXiv:1907.10853

%\bibitem[Eastwood et al.(2018)]{2018AJ....156...32E} Eastwood, M.~W., Anderson, M.~M., Monroe, R.~M., et al.\ 2018, {\it aj\/}, {\bf 156}, 32

%\bibitem[Ellingson and Kramer (2004)]{Memo28} Ellingson, S.~W., {Kramer}, W.~T. \ 2004, {\it Long Wavelength Array Memo (28)}

%\bibitem[George {et al.}(2018)] {article} George, M., Orchiston, W., Wielebinsk, R., {\it et al.} [2018], Journal of Astronomical History and Heritage, {\bf 21}, 37

%\bibitem[Hicks et al.(2012)]{2012PASP..124.1090H} Hicks, B.~C., Paravastu-Dalal, N., Stewart, K.~P., et al.\ 2012, {\it pasp\/}, {\bf 124}, 1090

%\bibitem[Koopmans {et al.}(2019)]{2019arXiv190804296K} Koopmans, L., Barkana, R., Bentum, M., {\it et al.} [2019], arXiv e-prints, arXiv:1908.04296

%\bibitem[Liu {et al.}(2013)]{2013PhRvD..87d3002L} Liu, A., Pritchard, J.~R., Tegmark, M., {\it et al.} [2013], {\bf 87}, 043002

%\bibitem[Philip {et al.}(2019)]{2019JAI.....850004P} Philip, L., Abdurashidova, Z., Chiang, H.~C., {\it et al.} [2019], Journal of Astronomical Instrumentation, {\bf 8}, 19500

%\bibitem[Pober {et al.}(2014)]{2014ApJ...782...66P} Pober, J.~C., Liu, A., Dillon, J.~S., {\it et al.} [2014], {\bf 782}, 66

%\bibitem[Ray et al.(2006)]{Memo35} Ray, P.~S., Ellingson, S.~W., Fisher R., {\it et al.} [2006] {\it Long Wavelength Array Memo (35)}

%\bibitem[Roger et al.(1999)]{1999A&AS..137....7R} Roger, R.~S., Costain, C.~H., Landecker, T.~L., et al.\ 1999, {\it aaps\/}, {\bf 137}, 7

%\bibitem[Weiler {et al.}(1988)]{1988A&A...195..372W} Weiler, K.~W., Johnston, K.~J., Simon, R.~S., {\it et al.} [1988], {\it aap\/}, {\bf 195}, 372

%\end{thebibliography}
\end{document}
