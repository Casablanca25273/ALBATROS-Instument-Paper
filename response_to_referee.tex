\documentclass[11pt]{article}
\usepackage{palatino}
\usepackage{graphicx}
\usepackage{url}
\usepackage{enumitem}
\usepackage{xcolor}
\usepackage[square,sort&compress]{natbib}
\usepackage[small,compact]{titlesec}
\usepackage[margin=0.85in]{geometry}
 
\setlength{\parskip}{0.07in}
\setlength{\bibsep}{0.05in}

\bibpunct{[}{]}{,}{n}{}{}

\makeatletter
\def\subsize{\@setsize\subsize{12pt}\xipt\@xipt}
\def\section{\@startsection {section}{1}{\z@}{1.0ex plus 
1ex minus .2ex}{.2ex plus .2ex}{\normalsize\bf}}
\def\subsection{\@startsection {subsection}{2}{\z@}{.2ex 
plus 1ex} {.2ex plus .2ex}{\subsize\bf}}
\makeatother

\def\scihi{{\sc SCI-HI}}
\def\albatros{ALBATROS}
\def\prizm{PRI$^{\rm Z}$M}

%%%%%%%%%%%%%%%%%%%%%%%%%%%%%%%%%%%%%%%%%%%%%%%%%%%%%%%%%%%%%%%%%%

\begin{document}

\noindent Dear Referees:

Thank you for your comments and constructive feedback on our paper.
Below are our responses to your comments, as well as descriptions of
corresponding manuscript changes.

\noindent \textcolor{blue}{1) Starting with the antenna, the authors
  mention this is not optimised for lower than 10MHz and another
  antenna needs to be developed but there is hardly any discussion on
  this. Just from the plots in figure 8, it is clear the performance
  of the antenna drops pretty rapidly below 30MHz, thus implying this
  antenna is pretty poorly matched at the primary frequencies of
  interest here. I would have liked to see at least some information
  about this antenna in the paper, perhaps even a plot of S11 which
  surely the authors have access to. }

We have included a new figure (9) that illustrates the simulated $S11$
for the antenna, in addition to an overplot of the median autospectrum
with a crude sky signal estimate given by $(1-S11)$ multiplied with a
$\nu^{-2.6}$ spectrum.

\noindent \textcolor{blue}{2) To develop an antenna to work
  effectively below 30MHz down to a few MHz, you will need to make it
  significantly larger, implying there will be major challenges in how
  you construct and deploy an antenna which may be 5m wide. If this is
  not the case and you are using an electrically short, albeit fat
  dipole, then the antenna is horribly mismatched at the lower
  frequencies. In my view this will have a major impact on 1-bit
  quantisation. How will that work when there is a 30dB or even 40dB
  slope across the band?}

\noindent \textcolor{blue}{3) Further on the 1-bit quantisation
  approach, would you not only need a better matched antenna but also
  an equalised analogue system to remove the extremely high slope from
  the synchrotron emission at low frequencies. This can be up to 20dB
  at these frequencies. Furthermore, despite talking about 1-bit
  quantisation, as far as I know the plots in the paper are all using
  4 or more bit? Has this been tested in the field or in the lab. I do
  not doubt it will work but it will need careful levelling and
  testing and ideally should be coupled with a noise injection signal
  for calibration.}

In response to both comments above, the SNAP processes the baseband
data using the full resolution of the ADCs (8-bit, with a 14-bit
version possibly in the future) and internally works at 18 bits.  It
is only after channelization that the data are downsampled, so
happily, concerns about cross-channel leakage (either due to RFI,
slopes across the band, antenna performance etc.) aren't affected by
the bit depth at which we save the data.  We have now clarified this
point in section 4.1 of the text and thank the referee for catching
the omission.

\noindent \textcolor{blue}{4) Regarding calibration of this
  instrument, there is virtually no mention of it. This is supposed to
  be a precision cosmology instrument and therefore I would have
  expected some more discussion on calibration. Even if not going down
  the route of an EDGES-style approach, having e.g. a Dicke switch in
  there would allow some measure of the scale and offset.}

First, we stress ALBATROS is not meant to be a precision cosmology
instrument---that will have to wait for future telescope generations.
Dicke switching, at least as part of regular operations, won't add
very much because we can calibrate gain variations off the Galactic
signal.  Each station saves total power autospectra, which have
enormous SNR and are repeated day-to-day.  We find the instrument is
very stable, with RMS gain fluctuations $<$0.6\% over several days,
and typical gain calibration errors of $<$0.04\% on 3-second
timescales based on the Galactic signal.  Any noise offsets will in
general not contribute to cross-correlations and so the chief concern
is to remove them when developing a Galactic model.  PRIZM, which has
several levels of calibration switching, is co-located with ALBATROS,
has similar beams, and operates in a similar frequency range ($>$30
MHz).  We can derive an approximate absolute calibration by comparing
the diurnal variation in Galactic signal (which is insensitive to any
offsets) between PRIZM and ALBATROS.  We have updated the text in
section~5 to reflect these numbers.

\noindent \textcolor{blue}{5) An important aspect of developing an
  autonomous instrument like this is timing. I can see the authors
  have thought about this and discussed it in the paper but have there
  been any experiments of running such systems for months on end and
  analysing the drifts. It seems like something that could be done
  fairly easily in the lab for extended periods of time. If there is
  no data to present on this, then some discussion about such possible
  plans would help.}

Unfortunately all we have are the preliminary lab tests.  The setup
was literally a few pieces of wire hanging off an LWA front end
sitting inside the lab, not even next to a window.  With just a few
minutes of that, we were able to synchronize the EW and NS
polarizations to about 0.1ns.  Our plan was to then go to the roof and
carry out proper tests (exactly the ones the referee suggests), but
unfortunately COVID shut down lab access just a couple of days after
the lab tests.  Similarly, we should have timing information from the
data on disk at Marion, but all science efforts were cancelled during
the 2020 voyage, so we have been unable to retrieve the data.  A
talented undergraduate is doing this work for his thesis, now that
labs are slowly opening back up, but the results will not be ready on
the timescale of resubmission.  We have added text in section 4.1
clarifying that the ORBCOMM and science data are recorded
simultaneously, and we can therefore iteratively improve the timing
calibration in post-processing.  Otherwise we cast ourselves on the
mercy of the journal and the referee to let us defer to a future paper
the analysis we would have loved to have been able to put here.

\noindent \textcolor{blue}{p.2: ``recent ground-based experiments''
  include a 1976 reference. Doesn't seem very recent.}

Changed wording from ``recent'' to ``subsequent.''

\noindent \textcolor{blue}{Could add 45 MHz efforts of Alvarez and
  Guzman.}

Since our discussion focuses on observations below 30 MHz, we feel
that these 45 MHz observations are beyond the scope of our
introductory text.

\noindent \textcolor{blue}{The statement ``Although this experimental
  list is not comprehensive, it does illustrate the dearth of
  information about ..'' is a bit self-defeating. A smaller list would
  illustrate an even larger dearth! I think it would be good to make
  this a complete list, given the fact that it is very short.}

Although we would love to one day write a review article (or read one;
to the best of our knowledge, such a review article doesn't presently
exist), compiling a comprehensive list that's 100\% accurate is a
non-trivial exercise.  We nevertheless agree that the text should be
revised, so we have now included a reference to the Global Sky Model,
which uses only the DRAO 10 MHz and 22 MHz maps as suitable model
inputs below 30 MHz.

\noindent \textcolor{blue}{para 2 is a bit long - split at ``The first step toward...''?}

Done.

\noindent \textcolor{blue}{These sections seem fine, but some more
  information about the LWA antenna below 10 MHz would be useful. The
  sentence ``LWA antenna is not optimized for observations below 10
  MHz...'' seems an understatement given that $<10$ MHz is outside the
  published frequency capability of the antenna. A more informed
  statement about sensitivity (or lack thereof would be useful). For
  example, the lack of Galactic Plane and RFI signal below 8 MHz in
  various plots (eg Figs 8, 11) looks more like there might be
  complete loss of sensitivity. Please clarify.}

We have included a new figure (9) that illustrates the simulated $S11$
for the antenna, in addition to an overplot of the median autospectrum
with a crude sky signal estimate given by $(1-S11)$ multiplied with a
$\nu^{-2.6}$ spectrum.

\noindent \textcolor{blue}{The review of 2-level correlator efficiency
  vs S/N ratio is interesting and a nice reminder to radio astronomers
  about the very different regime such a telescope could be
  operating. But the last paragraph of Section 4.2 dismisses the
  suggestion that the S/N ratio could be non-zero with a weird
  argument about Gaussian random fields. But the reality is if there
  is a massive continuum source in the field of view (Crab Nebula,
  Sun, Galactic Plane which all have structure on size scales $<1$
  deg), then this might indeed be a huge concern. The radio continuum
  sky is obviously not a Gaussian random field.}

We agree that the sky is certainly not a Gaussian random field, but to
get well away from the SNR$<$1 regime requires that a compact source
dominate the total power coming into the antenna, not just the power
in the visibility.  For most radio telescopes this would absolutely be
a concern, but the primary beam of ALBATROS is huge.  We already know
from e.g.\ EDGES that the total power is dominated by the Galactic
signal, and in PRIZM, the Crab (and typically the sun) are not obvious
in total power.  Due to ionospheric effects ALBATROS will not
generally be conducting daytime observing, and using the GSM and Crab
fluxes in the 10-80 MHz range, we expect the Crab is only a $\sim10$\%
effect in the total power, as long as the Galaxy is not self-absorbed.
Similarly, in Fig. 8, for a 100m baseline, the visibilities seen by
ALBATROS are only a few percent of the total power.  The text in
section 4.2 has been updated to include these details.

\noindent \textcolor{blue}{The waterfall plots are all fairly
  interesting, but of not much practical value without
  calibration. What is the frequency response of the system? How much
  fainter is the RFI in the cross-correlations than the
  autocorrelations? These simple questions can be answered with a
  simple model of how the system noise changes with frequency or
  utilising the nice Galactic plane drift scans in combination with a
  crude model.}

We prefer to defer proper calibration to a future paper since the
instrumentation presented here is intended as a proof of concept, and
crude calibration that isn't fully thought out is also of limited
practical value.  The frequency response of the system is roughly
illustrated with the new addition of figure~9.  To address the
question about RFI, we have added an extra phrase in section~5
explaining that the cross-spectrum RFI is about a factor of two
fainter than the autospectrum RFI.

\noindent \textcolor{blue}{Fig.8 caption: mention if the amplitude
  units are linear or log (the latter I think?). Also phase (radians I
  think?), although the latter has too much wrapping to be useful.}

We have added this information to the caption.

\noindent \textcolor{blue}{I suggest extracting some 1-d plots of
  amplitude and phase - these would be a great complement to the
  waterfall plots as they will contain features more difficult to see
  than on simple images.}

With the addition of figure 9, a 1D plot of the median autospectrum is
now present.  We have extracted a few 1D plots of the unwrapped phase
along the time axis (scaled by frequency), and we feel that the
results are not sufficiently interesting to add to the paper, although
we include the plot here for reference.
\begin{figure}[h]
  \vspace{-5pt}
  \begin{center}
    \includegraphics*[angle=0,width=0.5\textwidth]{Figures/unwrapped_phase_3freq.png}
  \end{center}
  \vspace{-5pt}
\end{figure}

\pagebreak

\noindent \textcolor{blue}{Perhaps mention (I didn't see) information
  about the frequency dependent beam size of the antenna, and the
  location of the phase centre of the interferometer (fixed at
  zenith?).}

We have added a sentence about the beam size variation (2.2--2.7
steradians over 5--100~MHz) and the phase center (at zenith) in
sections 3.1 and 3.2, respectively.

\noindent \textcolor{blue}{wind turbines have been investigated for
  potential use in several observatories before. Good idea potentially
  for the Marion site, but they are RFI sources and their motor parts
  or turbine blades may act as modulated reflectors or dielectrics.}

We agree, and investigating low-RFI solutions for wind turbines is
work in progress.  We are aware that low-RFI wind turbines are already
being developed for e.g.\ the ARIANNA experiment (arXiv:1903.01609).

\noindent\rule{\textwidth}{0.7pt}

Final note: we have added T.~M\'enard to the author list because he
contributed the $S11$ simulation in Figure~9.

\end{document}
